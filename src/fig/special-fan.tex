\begin{tikzpicture}
	
  \GraphInit[vstyle=Classic]

  \Vertex[Lpos=180]{$w$}
  \Vertex[Lpos=-90, a=0, d=4cm, L=$v_{1}$]{v1}
  \Vertex[Lpos=-90, a=25, d=4cm, L=$v_{2}$]{v2}
  \Vertex[Lpos=-90, a=50, d=4cm, L=$v_{3}$]{v3}
  \Vertex[Lpos=180, a=85, d=4cm, L=$v_{s - 1}$]{vsm1}
  \Vertex[Lpos=180, a=110, d=4cm, L=$v_{s}$]{vs}

  \Edge[style ={-}, label={$t_{1}$}, labelstyle={above, sloped}]({w})({v1})
  \Edge[style ={-, dashed}]({w})({v2})
  \Edge[style ={-}, label={$t_{2}$}, labelstyle={above, sloped}]({w})({v3})
  \Edge[style ={-}, label={$t_{s - 2}$},labelstyle={above, sloped}]({w})({vsm1})
  \Edge[style ={-}, label={$t_{s - 1}$},labelstyle={below, sloped}]({w})({vs})

  \Edge[style ={-,dashed,bend right}]({v3})({vsm1})

  \node[text width = 2cm, anchor=west, right=8pt] at (v1) {(no $t_{1}$)};
  \node[text width = 2cm, anchor=west, right=8pt] at (v2) {(no $t_{2}$)};
  \node[text width = 2cm, anchor=west, right=8pt] at (v3) {(no $t_{3}$)};
  \node[text width = 2cm, anchor=east, above=8pt] at (vsm1) {(no $t_{s - 1}$)};

\end{tikzpicture}

